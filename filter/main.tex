\documentclass[journal,12pt,twocolumn]{IEEEtran}
%
\usepackage{setspace}
\usepackage{gensymb}
\usepackage{xcolor}
\usepackage{caption}
%\usepackage{subcaption}
%\doublespacing
\singlespacing

\usepackage{graphicx}
%\usepackage{amssymb}
%\usepackage{relsize}
\usepackage[cmex10]{amsmath}
\usepackage{mathtools}
%\usepackage{amsthm}
%\interdisplaylinepenalty=2500
%\savesymbol{iint}
%\usepackage{txfonts}
%\restoresymbol{TXF}{iint}
%\usepackage{wasysym}
\usepackage{hyperref}
\usepackage{amsthm}
\usepackage{mathrsfs}
\usepackage{txfonts}
\usepackage{stfloats}
\usepackage{cite}
\usepackage{cases}
\usepackage{subfig}
%\usepackage{xtab}
\usepackage{longtable}
\usepackage{multirow}
%\usepackage{algorithm}
%\usepackage{algpseudocode}
%\usepackage{enumerate}
\usepackage{enumitem}
\usepackage{mathtools}
%\usepackage{iithtlc}
%\usepackage[framemethod=tikz]{mdframed}
\usepackage{listings}


%\usepackage{stmaryrd}


%\usepackage{wasysym}
%\newcounter{MYtempeqncnt}
\DeclareMathOperator*{\Res}{Res}
%\renewcommand{\baselinestretch}{2}
\renewcommand\thesection{\arabic{section}}
\renewcommand\thesubsection{\thesection.\arabic{subsection}}
\renewcommand\thesubsubsection{\thesubsection.\arabic{subsubsection}}

\renewcommand\thesectiondis{\arabic{section}}
\renewcommand\thesubsectiondis{\thesectiondis.\arabic{subsection}}
\renewcommand\thesubsubsectiondis{\thesubsectiondis.\arabic{subsubsection}}

%\renewcommand{\labelenumi}{\textbf{\theenumi}}
%\renewcommand{\theenumi}{P.\arabic{enumi}}

% correct bad hyphenation here
\hyphenation{op-tical net-works semi-conduc-tor}

\lstset{
	language=Python,
	frame=single, 
	breaklines=true,
	columns=fullflexible
}



\begin{document}
	%
	
	\theoremstyle{definition}
	\newtheorem{theorem}{Theorem}[section]
	\newtheorem{problem}{Problem}
	\newtheorem{proposition}{Proposition}[section]
	\newtheorem{lemma}{Lemma}[section]
	\newtheorem{corollary}[theorem]{Corollary}
	\newtheorem{example}{Example}[section]
	\newtheorem{definition}{Definition}[section]
	%\newtheorem{algorithm}{Algorithm}[section]
	%\newtheorem{cor}{Corollary}
	\newcommand{\BEQA}{\begin{eqnarray}}
		\newcommand{\EEQA}{\end{eqnarray}}
	\newcommand{\define}{\stackrel{\triangle}{=}}
	\newcommand{\myvec}[1]{\ensuremath{\begin{pmatrix}#1\end{pmatrix}}}
	\newcommand{\mydet}[1]{\ensuremath{\begin{vmatrix}#1\end{vmatrix}}}
	
	\bibliographystyle{IEEEtran}
	%\bibliographystyle{ieeetr}
	
	\providecommand{\nCr}[2]{\,^{#1}C_{#2}} % nCr
	\providecommand{\nPr}[2]{\,^{#1}P_{#2}} % nPr
	\providecommand{\mbf}{\mathbf}
	\providecommand{\pr}[1]{\ensuremath{\Pr\left(#1\right)}}
	\providecommand{\qfunc}[1]{\ensuremath{Q\left(#1\right)}}
	\providecommand{\sbrak}[1]{\ensuremath{{}\left[#1\right]}}
	\providecommand{\lsbrak}[1]{\ensuremath{{}\left[#1\right.}}
	\providecommand{\rsbrak}[1]{\ensuremath{{}\left.#1\right]}}
	\providecommand{\brak}[1]{\ensuremath{\left(#1\right)}}
	\providecommand{\lbrak}[1]{\ensuremath{\left(#1\right.}}
	\providecommand{\rbrak}[1]{\ensuremath{\left.#1\right)}}
	\providecommand{\cbrak}[1]{\ensuremath{\left\{#1\right\}}}
	\providecommand{\lcbrak}[1]{\ensuremath{\left\{#1\right.}}
	\providecommand{\rcbrak}[1]{\ensuremath{\left.#1\right\}}}
	\theoremstyle{remark}
	\newtheorem{rem}{Remark}
	\newcommand{\sgn}{\mathop{\mathrm{sgn}}}
	\providecommand{\abs}[1]{\left\vert#1\right\vert}
	\providecommand{\res}[1]{\Res\displaylimits_{#1}} 
	\providecommand{\norm}[1]{\lVert#1\rVert}
	\providecommand{\mtx}[1]{\mathbf{#1}}
	\providecommand{\mean}[1]{E\left[ #1 \right]}
	\providecommand{\fourier}{\overset{\mathcal{F}}{ \rightleftharpoons}}
	\providecommand{\ztrans}{\overset{\mathcal{Z}}{ \rightleftharpoons}}
	
	%\providecommand{\hilbert}{\overset{\mathcal{H}}{ \rightleftharpoons}}
	\providecommand{\system}{\overset{\mathcal{H}}{ \longleftrightarrow}}
	%\newcommand{\solution}[2]{\textbf{Solution:}{#1}}
	\newcommand{\solution}{\noindent \textbf{Solution: }}
	\providecommand{\dec}[2]{\ensuremath{\overset{#1}{\underset{#2}{\gtrless}}}}
	\numberwithin{equation}{section}
	%\numberwithin{equation}{subsection}
	%\numberwithin{problem}{subsection}
	%\numberwithin{definition}{subsection}
	\makeatletter
	\@addtoreset{figure}{problem}
	\makeatother
	
	\let\StandardThe~\thefigure
	%\renewcommand{\thefigure}{\theproblem.\arabic{figure}}
	\renewcommand{\thefigure}{\theproblem}
	
	
	%\numberwithin{figure}{subsection}
	
	\def\putbox#1#2#3{\makebox[0in][l]{\makebox[#1][l]{}\raisebox{\baselineskip}[0in][0in]{\raisebox{#2}[0in][0in]{#3}}}}
	\def\rightbox#1{\makebox[0in][r]{#1}}
	\def\centbox#1{\makebox[0in]{#1}}
	\def\topbox#1{\raisebox{-\baselineskip}[0in][0in]{#1}}
	\def\midbox#1{\raisebox{-0.5\baselineskip}[0in][0in]{#1}}
	
	\vspace{3cm}
	
	\title{ 
		%\logo{
			Digital Signal Processing
			%}
		%	\logo{Octave for Math Computing }
	}
	%\title{
		%	\logo{Matrix Analysis through Octave}{\begin{center}\includegraphics[scale=.24]{tlc}\end{center}}{}{HAMDSP}
		%}
	
	
	% paper title
	% can use linebreaks \\ within to get better formatting as desired
	%\title{Matrix Analysis through Octave}
	%
	%
	% author names and IEEE memberships
	% note positions of commas and nonbreaking spaces ( ~ ) LaTeX will not break
	% a structure at a ~ so this keeps an author's name from being broken across
	% two lines.
	% use \thanks{} to gain access to the first footnote area
	% a separate \thanks must be used for each paragraph as LaTeX2e's \thanks
	% was not built to handle multiple paragraphs
	%
	
	\author{ Mulugu Vishwanath Sharma %<-this  stops a space
		\\MA20BTECH11010
		\thanks{*The author is with the Department
			of Electrical Engineering, Indian Institute of Technology, Hyderabad
			502285 India e-mail:  gadepall@iith.ac.in.  All content in the manuscript is 
			released under GNU GPL.  Free to use for anything. }% <-this % stops a space
		%\thanks{J. Doe and J. Doe are with Anonymous University.}% <-this % stops a space
		%\thanks{Manuscript received April 19, 2005; revised January 11, 2007.}}
}
% note the % following the last \IEEEmembership and also \thanks - 
% these prevent an unwanted space from occurring between the last author name
% and the end of the author line. i.e., if you had this:
% 
% \author{....lastname \thanks{...} \thanks{...} }
%                     ^------------^------------^----Do not want these spaces!
%
% a space would be appended to the last name and could cause every name on that
% line to be shifted left slightly. This is one of those "LaTeX things". For
% instance, "\textbf{A} \textbf{B}" will typeset as "A B" not "AB". To get
% "AB" then you have to do: "\textbf{A}\textbf{B}"
% \thanks is no different in this regard, so shield the last } of each \thanks
% that ends a line with a % and do not let a space in before the next \thanks.
% Spaces after \IEEEmembership other than the last one are OK (and needed) as
% you are supposed to have spaces between the names. For what it is worth,
% this is a minor point as most people would not even notice if the said evil
% space somehow managed to creep in.



% The paper headers
%\markboth{Journal of \LaTeX\ Class Files,~Vol.~6, No.~1, January~2007}%
%{Shell \MakeLowercase{\textit{et al.}}: Bare Demo of IEEEtran.cls for Journals}
% The only time the second header will appear is for the odd numbered pages
% after the title page when using the twoside option.
% 
% *** Note that you probably will NOT want to include the author's ***
% *** name in the headers of peer review papers.                   ***
% You can use \ifCLASSOPTIONpeerreview for conditional compilation here if
% you desire.




% If you want to put a publisher's ID mark on the page you can do it like
% this:
%\IEEEpubid{0000--0000/00\$00.00~\copyright~2007 IEEE}
% Remember, if you use this you must call \IEEEpubidadjcol in the second
% column for its text to clear the IEEEpubid mark.



% make the title area
\maketitle

%\newpage

\tableofcontents

%\renewcommand{\thefigure}{\thesection.\theenumi}
%\renewcommand{\thetable}{\thesection.\theenumi}

\renewcommand{\thefigure}{\theenumi}
\renewcommand{\thetable}{\theenumi}

%\renewcommand{\theequation}{\thesection}


\bigskip

\begin{abstract}
This manual provides a simple introduction to digital signal processing.
\end{abstract}
\section{Software Installation}
Run the following commands
\begin{lstlisting}
sudo apt-get update
sudo apt-get install libffi-dev libsndfile1 python3-scipy  python3-numpy python3-matplotlib 
sudo pip install cffi pysoundfile 
\end{lstlisting}
\section{Digital Filter}
\begin{enumerate}[label=\thesection.\arabic*
,ref=\thesection.\theenumi]
\item
\label{prob:input}
Download the sound file from  
\begin{lstlisting}
	wget https://raw.githubusercontent.com/gadepall/ 
	EE1310/master/filter/codes/Sound_Noise.wav
\end{lstlisting}
%\href{http://tlc.iith.ac.in/img/sound/Sound_Noise.wav}{\url{http://tlc.iith.ac.in/img/sound/Sound_Noise.wav}}  
%in the link given below.
%\linebreak
\item
\label{prob:spectrogram}
You will find a spectrogram at \href{https://academo.org/demos/spectrum-analyzer}{\url{https://academo.org/demos/spectrum-analyzer}}. 
%\end{problem}
%%
%
%%\onecolumn
%%\input{.figs/fir}
%\begin{problem}
Upload the sound file that you downloaded in Problem \ref{prob:input} in the spectrogram  and play.  Observe the spectrogram. What do you find?
\\
%
\solution There are a lot of yellow lines between 440 Hz to 5.1 KHz.  These represent the synthesizer key tones. Also, the key strokes
are audible along with background noise.
% By observing spectrogram, it clearly shows that tonal frequency is under 4kHz. And above 4kHz only noise is present.
\item
\label{prob:output}
Write the python code for removal of out of band noise and execute the code.
\\
\solution
%\lstinputlisting{./Cancel_noise.py}
%\begin{figure}[h]
%\centering
%\includegraphics[width=\columnwidth]{enc_block_diag.png}
%\caption{}
%\label{fig:convolution encoder}
%\end{figure}
%\input{block_enc}
\item
  $The output of the python script in Problem \ref{prob:output} is the audio file Sound\_With\_ReducedNoise.wav. Play the file in the spectrogram in Problem \ref{prob:spectrogram}. What do you observe?$
\\
\solution The key strokes as well as background noise is subdued in the audio.  Also,  the signal is blank for frequencies above 5.1 kHz.

\end{enumerate}
\section{Difference Equation}
\begin{enumerate}[label=\thesection.\arabic*,ref=\thesection.\theenumi]
\item Let
\label{def:xn}
\begin{equation}
	x(n) = \cbrak{\underset{\uparrow}{1},2,3,4,2,1}
\end{equation}
Sketch $x(n)$.
\item Let
\begin{multline}
	\label{eq:iir_filter}
	y(n) + \frac{1}{2}y(n-1) = x(n) + x(n-2), 
	\\
	y(n) = 0, n < 0
\end{multline}
Sketch $y(n)$.  
\\
\solution The following code yields Fig. \ref{fig:xnyn}.
\begin{lstlisting}
	wget https://github.com/gadepall/EE1310/raw/master/filter/codes/xnyn.py
\end{lstlisting}
\begin{figure}[!ht]
	\begin{center}
		\includegraphics[width=\columnwidth]{/home/dhanush/Downloads/xnyn.png}
	\end{center}
	\captionof{figure}{}
	\label{fig:xnyn}	
\end{figure}
\item Repeat the above exercise using a C code.
\textbf{solution: } The following code is the implementation in C.

\begin{lstlisting}
	https://github.com/dhanushpittala11/EE3900-2022/blob/main/filter/codes/xn_yn.ipynb
\end{lstlisting}

\begin{figure}[!h]
	\begin{center}
		\includegraphics[width=\columnwidth]{/home/dhanush/Downloads/xnyn.png}
	\end{center}
	\captionof{figure}{}
	\label{fig:xnyn_2}	
\end{figure}

\end{enumerate}

\section{$Z$-transform}
\begin{enumerate}[label=\thesection.\arabic*]
\item The $Z$-transform of $x(n)$ is defined as
%
\begin{equation}
	\label{eq:z_trans}
	X(z)={\mathcal {Z}}\{x(n)\}=\sum _{n=-\infty }^{\infty }x(n)z^{-n}
\end{equation}
%
Show that
\begin{equation}
	\label{eq:shift1}
	{\mathcal {Z}}\{x(n-1)\} = z^{-1}X(z)
\end{equation}
and find
\begin{equation}
	{\mathcal {Z}}\{x(n-k)\} 
\end{equation}
\solution From \eqref{eq:z_trans},
\begin{align}
	{\mathcal {Z}}\{x(n-k)\} &=\sum _{n=-\infty }^{\infty }x(n-1)z^{-n}
	\\
	&=\sum _{n=-\infty }^{\infty }x(n)z^{-n-1} = z^{-1}\sum _{n=-\infty }^{\infty }x(n)z^{-n}
\end{align}
resulting in \eqref{eq:shift1}. Similarly, it can be shown that
%
\begin{equation}
	\label{eq:z_trans_shift}
	{\mathcal {Z}}\{x(n-k)\} = z^{-k}X(z)
\end{equation}
\item Obtain $X(z)$ for $x(n)$ defined in problem 
\ref{def:xn}.
\solution
\begin{equation}
	X(z) = \sum_{n=-\infty}^{\infty}x(n)z^{-n}
\end{equation}
But
\begin{equation}
	x(n) = \cbrak{1,2,3,4,2,1}
\end{equation}
so,
\begin{equation}
	X(z) = 1 + 2z^{-1} + 3z^{-2} + 4z^{-3} + 2z^{-4} + z^{-5}
\end{equation}
\item Find
%
\begin{equation}
	H(z) = \frac{Y(z)}{X(z)}
\end{equation}
%
from  \eqref{eq:iir_filter} assuming that the $Z$-transform is a linear operation.
\\
\solution  Applying \eqref{eq:z_trans_shift} in \eqref{eq:iir_filter},
\begin{align}
	Y(z) + \frac{1}{2}z^{-1}Y(z) &= X(z)+z^{-2}X(z)
	\\
	\implies \frac{Y(z)}{X(z)} &= \frac{1 + z^{-2}}{1 + \frac{1}{2}z^{-1}}
	\label{eq:freq_resp}
\end{align}
%
\item Find the Z transform of 
\begin{equation}
	\delta(n)
	=
	\begin{cases}
		1 & n = 0
		\\
		0 & \text{otherwise}
	\end{cases}
\end{equation}
and show that the $Z$-transform of
\begin{equation}
	\label{eq:unit_step}
	u(n)
	=
	\begin{cases}
		1 & n \ge 0
		\\
		0 & \text{otherwise}
	\end{cases}
\end{equation}
is
\begin{equation}
	U(z) = \frac{1}{1-z^{-1}}, \quad \abs{z} > 1
\end{equation}
\solution It is easy to show that
\begin{equation}
	\delta(n) \ztrans 1
\end{equation}
and from \eqref{eq:unit_step},
\begin{align}
	U(z) &= \sum _{n= 0}^{\infty}z^{-n}
	\\
	&=\frac{1}{1-z^{-1}}, \quad \abs{z} > 1
\end{align}
using the formula for the sum of an infinite geometric progression.
%
\item Show that 
\begin{equation}
	\label{eq:anun}
	a^nu(n) \ztrans \frac{1}{1-az^{-1}} \quad \abs{z} > \abs{a}
\end{equation}
%

\textbf{Solution: }
let
\begin{equation}
	f(n) = a^n u(n)
\end{equation}

\begin{equation}
	f(n) = 
	\left\{
	\begin{array}{lr}
		a^n, & \text{if } n>0\\
		0  , & \text{otherwise}
	\end{array}
	\right\}
\end{equation}

Now the Z- Transform of f(n) is
\begin{equation}
	F(z) = {\mathcal {Z}}\{f(n)\} = \sum_{n=-\infty}^{\infty} f(n)z^{-n}
\end{equation}

\begin{equation}
	F(z) = \sum_{n=0}^{\infty} a^{n}z^{-n}
\end{equation}
This forms a infinite Geometric Progression.
\begin{equation}
	F(z) = \frac{1}{1-az^{-1}} \text{ for } z<a.
\end{equation}
\item 
Let
\begin{equation}
	H\brak{e^{j \omega}} = H\brak{z = e^{j \omega}}.
\end{equation}
Plot $\abs{H\brak{e^{j \omega}}}$.Is it periodic? If so, find the period. Comment.  $H(e^{j \omega})$ is
known as the {\em Discret Time Fourier Transform} (DTFT) of $x(n)$.
\\
\solution 
\begin{equation}
	|H(e^{j \omega})| = \frac{|(1+cos(2\omega)- i sin(2 \omega))|}{|(1+\frac{1}{2}cos(\omega)|- i\frac{1}{2}sin(\omega))|}
\end{equation}

\begin{equation}
	|H(e^{j \omega})| = \frac{|2cos(\omega)|}{|\sqrt{\frac{5}{4} + cos(\omega)|}}
\end{equation}

\begin{equation}
	|H(e^{j \omega})| = \frac{|4cos(\omega)|}{|\sqrt{5+ 4cos(\omega)|}}
\end{equation}

we can see that the period of $H(e^{j \omega})$ is same as of $cos(\omega)$ which is $2\pi$. \\
Hence the period is $2\pi$.\\
The following code plots Fig.
\begin{lstlisting}
	wget https://raw.githubusercontent.com/gadepall/EE1310/master/filter/codes/dtft.py
\end{lstlisting}
\begin{figure}[!ht]
	\centering
	\includegraphics[width=\columnwidth]{/home/dhanush/Pictures/Screenshots/dtft.png}
	\caption{$\abs{H\brak{e^{j\omega}}}$}
	\label{fig:dtft}
\end{figure}

\item Express $x(n)$ in terms of $H\brak{e^{j \omega}}$.\\
\solution We have,
\begin{align}
	H(e^{\j\omega}) &= \sum_{k = -\infty}^{\infty}h(k)e^{-\j\omega k}
\end{align}
However,
\begin{align}
	\int_{-\pi}^{\pi}e^{\j\omega(n - k)}d\omega =
	\begin{cases}
		2\pi & n = k \\
		0 & \textrm{otherwise}
	\end{cases}
\end{align}
and so,
\begin{align}
	&\frac{1}{2\pi}\int_{-\pi}^{\pi}H(e^{\j\omega})e^{j\omega n}d\omega \\
	&= \frac{1}{2\pi}\sum_{k = -\infty}^{\infty}\int_{-\pi}^{\pi}h(k)e^{\j\omega(n - k)}d\omega \\
	&= \frac{1}{2\pi}2\pi h(n) = h(n)
\end{align}
which is known as the Inverse Discrete Fourier Transform. Thus,
\begin{align}
	h(n) &= \frac{1}{2\pi}\int_{-\pi}^{\pi}H(e^{\j\omega})e^{\j\omega n}d\omega \\
	&= \frac{1}{2\pi}\int_{-\pi}^{\pi}\frac{1 + e^{-2\j\omega}}{1 + \frac{1}{2}e^{-\j\omega}}e^{\j\omega n}d\omega
	\label{eq:idtft}
\end{align}
\end{enumerate}


\section{Impulse Response}
\begin{enumerate}[label=\thesection.\arabic*]
	\item Using long divsion, find
	\begin{equation}
		h(n) , n<5
	\end{equation}
	for $H(z)$ in \eqref{eq:freq_resp} \\
	\solution from \eqref{eq:freq_resp}
	\begin{equation}
		H(z) = \frac{1 + z^{-2}}{1 + \frac{1}{2}z^{-1}}
	\end{equation}
	\begin{center}
		$1 + \frac{1}{2}z^{-1}\overline{)1 + z^{-2}(} 2z^{-1}$\\
		${2z^{-1} + z^{-2}}$\\
		$1 + \frac{1}{2}z^{-1}\overline{)1-2z^{-1}(} -4$\\
		$-4 - 2z^{-1}$\\
		$\overline{5 + 0z^{-1}}$
	\end{center}
	Hence by long division will be 
	\begin{equation}
		H(z) = 2z^{-1} -4 + \frac{5}{1 + \frac{1}{2}z^{-1}}
	\end{equation}
	
	\item \label{prob:impulse_resp}
	Find an expression for $h(n)$ using $H(z)$, given that 
	%in Problem \ref{eq:ztransab} and \eqref{eq:anun}, given that
	\begin{equation}
		\label{eq:impulse_resp}
		h(n) \ztrans H(z)
	\end{equation}
	and there is a one to one relationship between $h(n)$ and $H(z)$. $h(n)$ is known as the {\em impulse response} of the
	system defined by \eqref{eq:iir_filter}.
	\\
	\solution From \eqref{eq:freq_resp},\\
	\begin{align}
		H(z) &= \frac{1}{1 + \frac{1}{2}z^{-1}} + \frac{ z^{-2}}{1 + \frac{1}{2}z^{-1}}
		\\
		\implies h(n) &= \brak{-\frac{1}{2}}^{n}u(n) + \brak{-\frac{1}{2}}^{n-2}u(n-2) \label{eq:h(n)}
	\end{align}
	using \eqref{eq:anun} and \eqref{eq:z_trans_shift}.
	\item Sketch $h(n)$. Is it bounded? Convergent? 
	\\
	\solution The following code plots Fig. \ref{fig:hn}.
	\begin{lstlisting}
		wget https://raw.githubusercontent.com/gadepall/EE1310/master/filter/codes/hn.py
	\end{lstlisting}
	\begin{figure}[!ht]
		\centering
		\includegraphics[width=\columnwidth]{/home/dhanush/Downloads/hn.png}
		\caption{$h(n)$ as the inverse of $H(z)$}
		\label{fig:hn}
	\end{figure}
	%
	\item The system with $h(n)$ is defined to be stable if
	\begin{equation}
		\sum_{n=-\infty}^{\infty}h(n) < \infty
	\end{equation}
	Is the system defined by \eqref{eq:iir_filter} stable for the impulse response in \eqref{eq:impulse_resp}?
	%
	
	\solution from \ref{eq:h(n)}
	\begin{equation}
		h(n) = \brak{-\frac{1}{2}}^{n}u(n) + \brak{-\frac{1}{2}}^{n-2}u(n-2) 
	\end{equation}
	then 
	\begin{equation}
		\sum_{n=-\infty}^{\infty}h(n) = \sum_{n=0}^{\infty} \brak{-\frac{1}{2}}^{n} + \sum_{n=2}^{\infty} \brak{-\frac{1}{2}}^{n-2}
	\end{equation}
	
	\begin{equation}
		\sum_{n=-\infty}^{\infty}h(n) = \frac{4}{3}
	\end{equation}
	
	since 
	\begin{equation}
		\sum_{n=-\infty}^{\infty}h(n) < \infty
	\end{equation}
	
	$h(n)$ is stable.
	
	\item 
	Compute and sketch $h(n)$ using 
	\begin{equation}
		\label{eq:iir_filter_h}
		h(n) + \frac{1}{2}h(n-1) = \delta(n) + \delta(n-2), 
	\end{equation}
	%
	This is the definition of $h(n)$.
	\\
	\solution The following code plots Fig. \ref{fig:hndef}. Note that this is the same as Fig. 
	\ref{fig:hn}. 
	%
	\begin{lstlisting}
		wget https://raw.githubusercontent.com/gadepall/EE1310/master/filter/codes/hndef.py
	\end{lstlisting}
	\begin{figure}[!ht]
		\centering
		\includegraphics[width=\columnwidth]{/home/dhanush/Pictures/Screenshots/hndef.png}
		\caption{$h(n)$ from the definition}
		\label{fig:hndef}
	\end{figure}
	%
	\item Compute 
	%
	\begin{equation}
		\label{eq:convolution}
		y(n) = x(n)*h(n) = \sum_{k=-\infty}^{\infty}x(k)h(n-k)
	\end{equation}
	%
	Comment. The operation in \eqref{eq:convolution} is known as
	{\em convolution}.
	%
	\\
	\solution The following code plots Fig. \ref{fig:ynconv}. Note that this is the same as 
	$y(n)$ in  Fig. 
	\ref{fig:xnyn}. 
	%
	\begin{lstlisting}
		wget https://raw.githubusercontent.com/gadepall/EE1310/master/filter/codes/ynconv.py
	\end{lstlisting}
	\begin{figure}[!ht]
		\centering
		\includegraphics[width=\columnwidth]{/home/dhanush/Downloads/yconv.png}
		\caption{$y(n)$ from the definition of convolution}
		\label{fig:ynconv}
	\end{figure}
	
	\item Show that
	\begin{equation}
		y(n) =  \sum_{k=-\infty}^{\infty}x(n-k)h(k)
	\end{equation}
	\solution from \ref{eq:convolution} ww know that
	\begin{equation}
		y(n) =  \sum_{k=-\infty}^{\infty}x(k)h(n-k) \label{eq:1st}
	\end{equation}
	
	now consider
	\begin{equation}
		t = n-k
	\end{equation}
	
	\ref{eq:1st} will transform into 
	\begin{equation}
		y(n) =  \sum_{n-t=-\infty}^{\infty}x(n-t)h(t) \label{eq:2nd}
	\end{equation}
	
	since n is finite and $-\infty < \infty$, \ref{eq:2nd} is equivalent to 
	
	\begin{equation}
		y(n) =  \sum_{t=-\infty}^{\infty}x(n-t)h(t) \label{eq:3nd}
	\end{equation}
	hence proved.
	
	
	\item Express the above convolution using a Teoplitz matrix.
	
	\text{5.9} 
	\begin{equation}
		y(n) = x(n)*h(n) = \sum_{k=-\infty}^{\infty}x(k)h(n-k)
	\end{equation}
	This can also be wrtten as a matrix-vector multiplication given by the expression,
	\begin{equation}
		\label{eq:conv_matrix_vec_mult}
		y = T\brack{h}*x
	\end{equation}
	$T\brack{h}$ is a Teoplitz matrix.
	
	\begin{align*}\\
		\
		\begin{pmatrix}
			h_1 & 0 & . & . & . & 0 \\
			h_2 & h_1 & . & . & . & 0 \\
			h_3 & h_2 & h_1 & . & . & 0 \\
			. & . & . & . & . & . \\
			h_{n-1} & h_{n-2} & h_{n-3} & . & . & 0\\
			h_{n} & h_{n-1} & h_{n-2} & . & . & h_1\\
			0 & h_{n} & h_{n-1} & h_{n-2} & . & h_2\\
			. & . & . & . & . & . \\
			0 & . & . & . & 0 & h_{n-1} \\
			0 & . & . & . & 0 & h_n \\
		\end{pmatrix}
		\
		\
		\begin{pmatrix}
			x_1 \\ x_2 \\ .\\.\\. \\ x_n
		\end{pmatrix}
		\
	\end{align*}
	
	\text{5.10}
	\item Show that
	\begin{equation}
		y(n) =  \sum_{n=-\infty}^{\infty}x(n-k)h(k)
	\end{equation}
	
	\begin{align}
		y(n) = \sum_{n=-\infty}^{\infty}x(k)h(n-k)
	\end{align}
	Taking k = n-k
	\begin{align}
		y(n) =  \sum_{n=-\infty}^{\infty}x(n-k)h(k)
	\end{align}
\end{enumerate}

%
\section{DFT and FFT}
\begin{enumerate}[label=\thesection.\arabic*]
	\item
	Compute
	\begin{equation}
		X(k) \define \sum _{n=0}^{N-1}x(n) e^{-j2\pi kn/N}, \quad k = 0,1,\dots, N-1
	\end{equation}
	and $H(k)$ using $h(n)$.\\
	\solution The following code plots $X(k)$ and $H(k)$.
	\begin{lstlisting}
		wget
		https://github.com/dhanushpittala11/EE3900-2022/blob/main/filter/codes/XkHk_dft.ipynb
	\end{lstlisting}
	
	\begin{figure}[!ht]
		\centering
		\includegraphics[width=\columnwidth]{/media/dhanush/New Volume/XkYk_Dft.png}
		\caption{}
		\label{fig:Hk_Xk}
	\end{figure}
	
	\item Compute 
	\begin{equation}
		Y(k) = X(k)H(k)
	\end{equation}
	\solution The following code plots $Y(k)$.
	\begin{lstlisting}
		wget
		https://github.com/dhanushpittala11/EE3900-2022/blob/main/filter/codes/Y_k.ipynb
	\end{lstlisting}
	
	\begin{figure}[!ht]
		\centering
		\includegraphics[width=\columnwidth]{/home/dhanush/Downloads/Yk.png}
		\caption{}
		\label{fig:Yk}
	\end{figure}
	
	\item Compute
	\begin{equation}
		y\brak{n}={\frac {1}{N}}\sum _{k=0}^{N-1}Y\brak{k}\cdot e^{j 2\pi kn/N},\quad n = 0,1,\dots, N-1
	\end{equation}
	\\
	\solution The following code plots Fig. \ref{fig:ynconv}. Note that this is the same as 
	$y(n)$ in  Fig. 
	\ref{fig:xnyn}. 
	%
	\begin{lstlisting}
		wget https://github.com/dhanushpittala11/EE3900-2022/blob/main/filter/codes/yndft.ipynb
	\end{lstlisting}
	\begin{figure}[!ht]
		\centering
		\includegraphics[width=\columnwidth]{/home/dhanush/Downloads/yndft}
		\caption{$y(n)$ from the DFT}
		\label{fig:yndft}
	\end{figure}
	
	\item Repeat the previous exercise by computing $X(k), H(k)$ and $y(n)$ through FFT and 
	IFFT.
	\solution The following code plots $X(n)$, $H(n)$ and $y(n)$ by fft.
	\begin{lstlisting}
		wget
		https://github.com/dhanushpittala11/EE3900-2022/blob/main/filter/codes/XkHk_fft.ipynb
	\end{lstlisting}
	
	\begin{figure}[!ht]
		\centering
		\includegraphics[width=\columnwidth]{/home/dhanush/Downloads/XkHk_Yk.png}
		\caption{$X(k)~$, $H(k)$ and $y(n)$ from fft and IFFT}
		\label{fig:Xk_Hk_yn}
	\end{figure}
	
	\item compare $y(n)$ obtained in \ref{fig:yndft} and \ref{fig:xnyn_2} and IFFT.\\
	\solution 
	The below code plots the plot of both the y(n)'s 
	
	\begin{lstlisting}
		wget
		https://github.com/dhanushpittala11/EE3900-2022/blob/main/filter/codes/compare.ipynb
	\end{lstlisting}
	
	The plot will be \ref{fig:2y(n)'s}
	
	\begin{figure}[!ht]
		\centering
		\includegraphics[width=\columnwidth]{/media/dhanush/New Volume/compare1.png}
		\caption{y(n)'s from \ref{fig:yndft} and \ref{fig:xnyn_2}}
		\label{fig:2y(n)'s}
	\end{figure}
	
	The following code compares $y(n)$'s obtained by \ref{fig:yndft}, \ref{fig:xnyn_2} and IFFT.
	\begin{lstlisting}
		wget
		https://github.com/dhanushpittala11/EE3900-2022/blob/main/filter/codes/compare2.ipynb
	\end{lstlisting}
	
	The plot is \ref{fig:3y(n)'s}
	
	\begin{figure}[!ht]
		\centering
		\includegraphics[width=\columnwidth]{/media/dhanush/New Volume/compare2.png}
		\caption{y(n)'s from \ref{fig:yndft} and \ref{fig:xnyn_2} and IFFT}
		\label{fig:3y(n)'s}
	\end{figure}
	
	\item Wherever possible, express all the above equations as matrix equations.\\
	\solution \\
	\[
	\begin{array}{lc}
		x = 
		\left(\begin{array}{@{}ccc@{}}
			1 \\
			2 \\
			3 \\
			4 \\
			2 \\
			1 \\
			0 \\
			. \\
			. \\
			. \\
			0 \\
		\end{array}\right) \\[15pt]
	\end{array}
	\]
	
	\[
	\begin{array}{lc}
		J = 
		\left(\begin{array}{@{}ccc@{}}
			0 \\
			e^{\frac{-2\pi j(1)k}{N}} \\
			e^{\frac{-2\pi j(2)k}{N}} \\
			. \\
			. \\
			. \\
			e^{\frac{-2\pi j(N-1)k}{N}} \\
		\end{array}\right) \\[15pt]
	\end{array}
	\]
	\begin{equation}
		X(k) = x^{T} J
	\end{equation}
	
	\[
	\begin{array}{lc}
		h = 
		\left(\begin{array}{@{}ccc@{}}
			h[0] \\
			h[1] \\
			. \\
			. \\
			. \\
			h[N-1] \\
		\end{array}\right) \\[15pt]
	\end{array}
	\]
	
	\begin{equation}
		H(k) = h^{T} J
	\end{equation}
	
	\[
	\begin{array}{lc}
		y = 
		\left(\begin{array}{@{}ccc@{}}
			h[0] x[0] \\
			h[1] x[1] \\
			. \\
			. \\
			. \\
			h[N-1] x[N-1] \\
		\end{array}\right) \\[15pt]
	\end{array}
	\]
	
	\begin{equation}
		Y(k) = y^{T} J
	\end{equation} 
	
\end{enumerate}



\section{FFT}
\subsection{Definitions}
\begin{enumerate}[label=\arabic*.,ref=\thesection.\theenumi]
	\numberwithin{equation}{section}
	\item The DFT of $x(n)$ is given by
	\begin{align}
		X(k) \triangleq \sum_{n=0}^{N-1} x(n) e^{-j 2 \pi k n / N}, \quad k=0,1, \ldots, N-1
	\end{align}
	\item Let 
	\begin{align}
		W_{N} = e^{-j2\pi/N} 
	\end{align}
	Then the $N$-point {\em DFT matrix} is defined as 
	\begin{align}
		\vec{F}_{N} = \sbrak{W_{N}^{mn}}
	\end{align}
	where $W_{N}^{mn}$ are the elements of $\vec{F}_{N}$.
	\item Let 
	\begin{align}
		\vec{I}_4 = \myvec{\vec{e}_4^{1} &\vec{e}_4^{2} &\vec{e}_4^{3} &\vec{e}_4^{4} }
	\end{align}
	be the $4\times 4$ identity matrix.  Then the 4 point {\em DFT permutation matrix} is defined as 
	\begin{align}
		\vec{P}_4 = \myvec{\vec{e}_4^{1} &\vec{e}_4^{3} &\vec{e}_4^{2} &\vec{e}_4^{4} }
	\end{align}
	\item The 4 point {\em DFT diagonal matrix} is defined as 
	\begin{align}
		\vec{D}_4 = diag\myvec{W_{N}^{0} & W_{N}^{1} & W_{N}^{2} & W_{N}^{3}}
	\end{align}
\end{enumerate}
\subsection{Problems}
\begin{enumerate}[label=\arabic*.,ref=\thesection.\theenumi]
	\numberwithin{equation}{section}
	\item Show that 
	\begin{equation}
		W_{N}^{2}=W_{N/2}
	\end{equation}
	
	\solution We know that.
	\begin{equation}
		W_{N} = e^{-j2\pi/N} 
	\end{equation}
	
	Then 
	\begin{align}
		W_{N/2} = e^{-2*j2\pi/N} \\
		W_{N/2} = {W_{N}}^{2}
	\end{align}
	Hence Proved.
	
	\item Show that 
	\begin{equation}
		\vec{F}_{4}=
		\begin{bmatrix}
			\vec{I}_{2} & \vec{D}_{2} \\
			\vec{I}_{2} & -\vec{D}_{2}
		\end{bmatrix}
		\begin{bmatrix}
			\vec{F}_{2} & 0 \\
			0 & \vec{F}_{2}
		\end{bmatrix}
		\vec{P}_{4}
	\end{equation}
	\solution Observe that for $n \in \mathbb{N}$, $W_4^{4n} = 1$ and $W_4^{4n + 2} = -1$. Using \eqref{eq:n-2},
	\begin{align}
		\vec{D}_2\vec{F}_2 &= \myvec{W_{4}^{0} & 0 \\ 0 & W_{4}^{1}}\myvec{W_{2}^{0} & W_{2}^{0} \\ W_{2}^{0} & W_{2}^{1}} \\
		&= \myvec{W_{4}^{0} & 0 \\ 0 & W_{4}^{1}}\myvec{W_{4}^{0} & W_{4}^{0} \\ W_{4}^{0} & W_{4}^{2}} \\
		&= \myvec{W_{4}^{0} & W_{4}^{0} \\ W_{4}^{1} & W_{4}^{3}} \label{eq:fft-df1} \\
		\implies -\vec{D}_2\vec{F}_2 &= \myvec{W_{4}^{2} & W_{4}^{6} \\ W_{4}^{3} & W_{4}^{9}} \label{eq:fft-df2}
	\end{align}
	and
	\begin{align}
		\vec{F}_2 &= \myvec{W_{2}^{0} & W_{2}^{0} \\ W_{2}^{0} & W_{2}^{1}} \\
		&= \myvec{W_{4}^{0} & W_{4}^{0} \\ W_{4}^{0} & W_{4}^{2}}
	\end{align}
	Hence,
	\begin{align}
		\vec{W}_4 &= \myvec{W_{4}^{0} & W_{4}^{0} & W_{4}^{0} & W_{4}^{0} \\
			W_{4}^{0} & W_{4}^{2} & W_{4}^{1} & W_{4}^{3} \\
			W_{4}^{0} & W_{4}^{4} & W_{4}^{2} & W_{4}^{6} \\
			W_{4}^{0} & W_{4}^{6} & W_{4}^{3} & W_{4}^{9} 
		} \label{eq:fft-permutation} \\
		&= \myvec{\vec{I}_2\vec{F}_2 & \vec{D}_2{F}_2 \\ \vec{I}_2\vec{F}_2 & -\vec{D}_2{F}_2} \\
		&= \myvec{\vec{I}_2 & \vec{D}_2 \\ \vec{I}_2 & \vec{D}_2}\myvec{\vec{F}_2 & 0 \\ 0 & \vec{F}_2}
		\label{eq:ifd}
	\end{align}
	Multiplying \eqref{eq:ifd} by $\vec{P}_4$ on both sides, and noting that $\vec{W}_4\vec{P}_4 = \vec{F}_4$ gives us.
	
	\item Show that 
	\begin{equation}
		\vec{F}_{N}=
		\begin{bmatrix}
			\vec{I}_{N/2} & \vec{D}_{N/2} \\
			\vec{I}_{N/2} & -\vec{D}_{N/2}
		\end{bmatrix}
		\begin{bmatrix}
			\vec{F}_{N/2} & 0 \\
			0 & \vec{F}_{N/2}
		\end{bmatrix}
		\vec{P}_{N}
	\end{equation}
	\solution Observe that for even $N$ and letting $\vec{f}_N^i$ denote the $i^{\text{th}}$ column of $\vec{F}_N$, from \eqref{eq:fft-df1} and \eqref{eq:fft-df2},
	\begin{align}
		\myvec{\vec{D}_{N/2}\vec{F}_{N/2} \\ -\vec{D}_{N/2}\vec{F}_{N/2}} = \myvec{\vec{f}_N^{2} & \vec{f}_N^{4} & \ldots & \vec{f}_N^{N}}
	\end{align}
	and
	\begin{align}
		\myvec{\vec{I}_{N/2}\vec{F}_{N/2} \\ \vec{I}_{N/2}\vec{F}_{N/2}} = \myvec{\vec{f}_N^{1} & \vec{f}_N^{3} & \ldots & \vec{f}_N^{N - 1}}
	\end{align}
	Thus,
	\begin{align}
		&\myvec{\vec{I}_2\vec{F}_2 & \vec{D}_2\vec{F}_2 \\ \vec{I}_2\vec{F}_2 & -\vec{D}_2\vec{F}_2} = \myvec{\vec{I}_{N/2} & \vec{D}_{N/2} \\ \vec{I}_{N/2} & -\vec{D}_{N/2}}\myvec{\vec{F}_{N/2} & 0 \\ 0 & \vec{F}_{N/2}} \nonumber \\
		&= \myvec{\vec{f}_N^{1} & \ldots & \vec{f}_N^{N - 1} & \vec{f}_N^{2} & \ldots & \vec{f}_N^{N}}
	\end{align}
	and so,
	\begin{align}
		&\myvec{\vec{I}_{N/2} & \vec{D}_{N/2} \\ \vec{I}_{N/2} & -\vec{D}_{N/2}}\myvec{\vec{F}_{N/2} & 0 \\ 0 & \vec{F}_{N/2}}\vec{P}_{N} \nonumber \\
		&= \myvec{\vec{f}_N^{1} & \vec{f}_N^{2} & \ldots & \vec{f}_N^{N}} = \vec{F}_N
	\end{align}
	
	\item Find 
	\begin{align}
		\vec{P}_6 \vec{x}
	\end{align}
     \solution We have,
     \begin{align}
     	\vec{P}_4\vec{x} = \myvec{\vec{e}_4^1 & \vec{e}_4^3 & \vec{e}_4^2 & \vec{e}_4^4}\myvec{x(0)\\x(1)\\x(2)\\x(3)} = \myvec{x(0)\\x(2)\\x(1)\\x(3)}
     	\label{eq:x-permute}
     \end{align}
 
 
	\item Show that 
	\begin{align}
		\vec{X} = \vec{F}_N \vec{x}
		\label{eq:dft-mat-def}
	\end{align}
	where $\vec{x}, \vec{X}$ are the vector representations of $x(n), X(k)$ respectively.
	
	\solution Writing the terms of $X$, 
	\begin{align}
		X(0) &= x(0) + x(1) + \ldots + x(N - 1) \\
		X(1) &= x(0) + x(1)e^{-\frac{\j2\pi}{N}} + \ldots + \nonumber \\
		&+ x(N - 1)e^{-\frac{\j2(N - 1)\pi}{N}} \\
		&\vdots \nonumber \\
		X(N - 1) &= x(0) + x(1)e^{-\frac{\j2(N - 1)\pi}{N}} + \ldots + \nonumber \\
		&+ x(N - 1)e^{-\frac{\j2(N - 1)(N - 1)\pi}{N}}	
	\end{align}
	Clearly, the term in the $m^{\text{th}}$ row and $n^{\text{th}}$ column is given by ($0 \leq m \leq N - 1$ and $0 \leq n \leq N - 1$) 
	\begin{align}
		T_{mn} = x(n)e^{-\frac{\j2mn\pi}{N}} 
	\end{align}
	and so, we can represent each of these terms as a matrix product
	\begin{align}
		\vec{X} = \vec{F}_N\vec{x}
	\end{align}
	where $\vec{F}_N = \sbrak{e^{-\frac{-\j2mn\pi}{N}}}_{mn}$ for $0 \leq m \leq N - 1$ and $0 \leq n \leq N - 1$. 
	
	
	
	\item Derive the following Step-by-step visualisation  of
	8-point FFTs into 4-point FFTs and so on
	\begin{equation}
		\begin{bmatrix}
			X(0) \\ 
			X(1) \\ 
			X(2) \\ 
			X(3)
		\end{bmatrix}
		=
		\begin{bmatrix}
			X_{1}(0) \\ 
			X_{1}(1)\\ 
			X_{1}(2)\\
			X_{1}(3)\\
		\end{bmatrix}
		+
		\begin{bmatrix}
			W^{0}_{8} & 0 & 0 & 0\\
			0 & W^{1}_{8} & 0 & 0\\
			0 & 0 & W^{2}_{8} & 0\\
			0 & 0 & 0 & W^{3}_{8}
		\end{bmatrix}
		\begin{bmatrix}
			X_{2}(0) \\ 
			X_{2}(1) \\ 
			X_{2}(2) \\
			X_{2}(3)
		\end{bmatrix}
	\end{equation}
	\begin{equation}
		\begin{bmatrix}
			X(4) \\ 
			X(5) \\ 
			X(6) \\ 
			X(7)
		\end{bmatrix}
		=
		\begin{bmatrix}
			X_{1}(0) \\ 
			X_{1}(1)\\ 
			X_{1}(2)\\
			X_{1}(3)\\
		\end{bmatrix}
		-
		\begin{bmatrix}
			W^{0}_{8} & 0 & 0 & 0\\
			0 & W^{1}_{8} & 0 & 0\\
			0 & 0 & W^{2}_{8} & 0\\
			0 & 0 & 0 & W^{3}_{8}
		\end{bmatrix}
		\begin{bmatrix}
			X_{2}(0) \\ 
			X_{2}(1) \\ 
			X_{2}(2) \\
			X_{2}(3)
		\end{bmatrix}
	\end{equation}
	4-point FFTs into 2-point FFTs
	\begin{equation}
		\begin{bmatrix}
			X_{1}(0) \\ 
			X_{1}(1)\\ 
		\end{bmatrix}
		=
		\begin{bmatrix}
			X_{3}(0) \\ 
			X_{3}(1)\\ 
		\end{bmatrix}
		+
		\begin{bmatrix}
			W^{0}_{4} & 0\\
			0 & W^{1}_{4}
		\end{bmatrix}
		\begin{bmatrix}
			X_{4}(0) \\ 
			X_{4}(1) \\ 
		\end{bmatrix}
	\end{equation}
	\begin{equation}
		\begin{bmatrix}
			X_{1}(2) \\ 
			X_{1}(3)\\ 
		\end{bmatrix}
		=
		\begin{bmatrix}
			X_{3}(0) \\ 
			X_{3}(1)\\ 
		\end{bmatrix}
		-
		\begin{bmatrix}
			W^{0}_{4} & 0\\
			0 & W^{1}_{4}
		\end{bmatrix}
		\begin{bmatrix}
			X_{4}(0) \\ 
			X_{4}(1) \\ 
		\end{bmatrix}
	\end{equation}
	\begin{equation}
		\begin{bmatrix}
			X_{2}(0) \\ 
			X_{2}(1)\\ 
		\end{bmatrix}
		=
		\begin{bmatrix}
			X_{5}(0) \\ 
			X_{5}(1)\\ 
		\end{bmatrix}
		+
		\begin{bmatrix}
			W^{0}_{4} & 0\\
			0 & W^{1}_{4}
		\end{bmatrix}
		\begin{bmatrix}
			X_{6}(0) \\ 
			X_{6}(1) \\ 
		\end{bmatrix}
	\end{equation}
	\begin{equation}
		\begin{bmatrix}
			X_{2}(2) \\ 
			X_{2}(3)\\ 
		\end{bmatrix}
		=
		\begin{bmatrix}
			X_{5}(0) \\ 
			X_{5}(1)\\ 
		\end{bmatrix}
		-
		\begin{bmatrix}
			W^{0}_{4} & 0\\
			0 & W^{1}_{4}
		\end{bmatrix}
		\begin{bmatrix}
			X_{6}(0) \\ 
			X_{6}(1) \\ 
		\end{bmatrix}
	\end{equation}
	\begin{equation}
		P_{8}
		\begin{bmatrix}
			x(0) \\ 
			x(1) \\ 
			x(2) \\ 
			x(3) \\ 
			x(4) \\ 
			x(5) \\
			x(6) \\
			x(7)
		\end{bmatrix}
		= 
		\begin{bmatrix}
			x(0) \\ 
			x(2) \\ 
			x(4) \\ 
			x(6) \\
			x(1) \\ 
			x(3) \\ 
			x(5) \\
			x(7)
		\end{bmatrix}
	\end{equation}
	\begin{equation}
		P_{4}
		\begin{bmatrix}
			x(0) \\ 
			x(2) \\ 
			x(4) \\ 
			x(6) \\
		\end{bmatrix}
		= 
		\begin{bmatrix}
			x(0) \\ 
			x(4) \\ 
			x(2) \\
			x(6)
		\end{bmatrix}
	\end{equation}
	\begin{equation}
		P_{4}
		\begin{bmatrix}
			x(1) \\ 
			x(3) \\ 
			x(5) \\
			x(7)
		\end{bmatrix}
		= 
		\begin{bmatrix}
			x(1) \\ 
			x(5) \\ 
			x(3) \\ 
			x(7) \\
		\end{bmatrix}
	\end{equation}
	Therefore,
	\begin{equation}
		\begin{bmatrix}
			X_{3}(0) \\ 
			X_{3}(1)\\ 
		\end{bmatrix}
		= F_{2}
		\begin{bmatrix}
			x(0) \\ 
			x(4) \\ 
		\end{bmatrix}
	\end{equation}
	\begin{equation}
		\begin{bmatrix}
			X_{4}(0) \\ 
			X_{4}(1)\\ 
		\end{bmatrix}
		= F_{2}
		\begin{bmatrix}
			x(2) \\ 
			x(6) \\ 
		\end{bmatrix}
	\end{equation}
	\begin{equation}
		\begin{bmatrix}
			X_{5}(0) \\ 
			X_{5}(1)\\ 
		\end{bmatrix}
		= F_{2}
		\begin{bmatrix}
			x(1) \\ 
			x(5) \\ 
		\end{bmatrix}
	\end{equation}
	\begin{equation}
		\begin{bmatrix}
			X_{6}(0) \\ 
			X_{6}(1)\\ 
		\end{bmatrix}
		= F_{2}
		\begin{bmatrix}
			x(3) \\ 
			x(7) \\ 
		\end{bmatrix}
	\end{equation}
	\solution We write out the values of performing an 8-point FFT on $\vec{x}$ as follows.
	\begin{align}
		X(k) &= \sum_{n = 0}^{7}x(n)e^{-\frac{\j2kn\pi}{8}} \\
		&= \sum_{n = 0}^{3}\brak{x(2n)e^{-\frac{\j2kn\pi}{4}} + e^{-\frac{\j2k\pi}{8}}x(2n + 1)e^{-\frac{\j2kn\pi}{4}}} \\
		&= X_1(k) + e^{-\frac{\j2k\pi}{4}}X_2(k) 
	\end{align}
	where $\vec{X}_1$ is the 4-point FFT of the even-numbered terms and $\vec{X}_2$ is the 4-point FFT of the odd numbered terms. Noticing that for $k \geq 4$,
	\begin{align}
		X_1(k) &= X_1(k - 4) \\
		e^{-\frac{\j2k\pi}{8}} &= -e^{-\frac{\j2(k - 4)\pi}{8}}
	\end{align}
	we can now write out $X(k)$ in matrix form as in \eqref{eq:8-low} and \eqref{eq:8-high}. We also need to solve the two 4-point FFT terms so formed.
	\begin{align}
		X_1(k) &= \sum_{n = 0}^{3}x_1(n)e^{-\frac{\j2kn\pi}{8}} \\
		&= \sum_{n = 0}^{1}\brak{x_1(2n)e^{-\frac{\j2kn\pi}{4}} + e^{-\frac{\j2k\pi}{8}}x_2(2n + 1)e^{-\frac{\j2kn\pi}{4}}} \\
		&= X_3(k) + e^{-\frac{\j2k\pi}{4}}X_4(k) 
	\end{align}
	using $x_1(n) = x(2n)$ and $x_2(n) = x(2n + 1)$. Thus we can write the 2-point FFTs
	\begin{align}
		\begin{bmatrix}
			X_{3}(0) \\ 
			X_{3}(1)\\ 
		\end{bmatrix}
		= F_{2}
		\begin{bmatrix}
			x(0) \\ 
			x(4) \\ 
		\end{bmatrix} \\
		\begin{bmatrix}
			X_{4}(0) \\ 
			X_{4}(1)\\ 
		\end{bmatrix}
		= F_{2}
		\begin{bmatrix}
			x(2) \\ 
			x(6) \\ 
		\end{bmatrix}
	\end{align}
	Using a similar idea for the terms $X_2$, 
	\begin{align}
		\begin{bmatrix}
			X_{5}(0) \\ 
			X_{5}(1)\\ 
		\end{bmatrix}
		= F_{2}
		\begin{bmatrix}
			x(1) \\ 
			x(5) \\ 
		\end{bmatrix} \\
		\begin{bmatrix}
			X_{6}(0) \\ 
			X_{6}(1)\\ 
		\end{bmatrix}
		= F_{2}
		\begin{bmatrix}
			x(3) \\ 
			x(7) \\ 
		\end{bmatrix}
	\end{align}
	But observe that from \eqref{eq:x-permute},
	\begin{align}
		\vec{P}_8\vec{x} &= \myvec{\vec{x}_1\\\vec{x}_2} \\
		\vec{P}_4\vec{x}_1 &= \myvec{\vec{x}_3\\\vec{x}_4} \\ 
		\vec{P}_4\vec{x}_2 &= \myvec{\vec{x}_5\\\vec{x}_6}
	\end{align}
	where we define $x_3(k) = x(4k)$, $x_4(k) = x(4k + 2)$, $x_5(k) = x(4k + 1)$, and $x_6(k) = x(4k + 3)$ for $k = 0, 1$.
	
	

	\begin{align}
		\vec{x} = \myvec{1\\2\\3\\4\\2\\1}
		\label{eq:equation1}
	\end{align}
\solution\\
The code below gives the answer
\begin{figure}[!ht]
	\centering
	\includegraphics[width=\columnwidth]{/home/dhanush/Downloads/7_11.png}
	
\end{figure}


	compte the DFT using 
	\eqref{eq:dft-mat-def}
	
	
	
	\item Repeat the above exercise using
	\eqref{eq:fft-mat-def}
	\item Write a C program to compute the 8-point FFT. 
	\solution The following code calculates the 8-point fft of $x(n)$ in \ref{def:xn}
	\begin{lstlisting}
		wget
		https://github.com/dhanushpittala11/EE3900-2022/blob/main/filter/codes/8_point_FFT.c
	\end{lstlisting}
	The result is \ref{fig:8_point_FFT}
	\begin{figure}[!htb]
		\includegraphics[width=\columnwidth]{/home/dhanush/Downloads/8pointFFT.png}
		\caption{8-Point FFT}
		\label{fig:8_point_FFT}
		
	\end{figure}
	
\end{enumerate}









\section{Exercises}

Answer the following questions by looking at the python code in Problem \ref{prob:output}.
\begin{enumerate}[label=\thesection.\arabic*]
	\item
	The command
	\begin{lstlisting}
		output_signal = signal.lfilter(b, a, input_signal)
	\end{lstlisting}
	in Problem \ref{prob:output} is executed through the following difference equation
	\begin{equation}
		\label{eq:iir_filter_gen}
		\sum _{m=0}^{M}a\brak{m}y\brak{n-m}=\sum _{k=0}^{N}b\brak{k}x\brak{n-k}
	\end{equation}
	%
	where the input signal is $x(n)$ and the output signal is $y(n)$ with initial values all 0. Replace
	\textbf{signal.filtfilt} with your own routine and verify.
	%
	\solution 
	\begin{lstlisting}
		https://github.com/dhanushpittala11/EE3900-2022/blob/main/filter/codes/8_1.ipynb
	\end{lstlisting}
	\item Repeat all the exercises in the previous sections for the above $a$ and $b$.
	\solution
	For the given values, the difference equation is
	\begin{align}
		&y(n) - \brak{4.44}y(n - 1) + \brak{8.78}y(n - 2) \nonumber \\
		&- \brak{9.93}y(n - 3) + \brak{6.90}y(n - 4) \nonumber \\
		&- \brak{2.93}y(n - 5) \nonumber + \brak{0.70}y(n - 6) \nonumber \\
		&- \brak{0.07}y(n - 7) = \brak{5.02 \times 10^{-5}}x(n) \nonumber \\
		&+ \brak{3.52 \times 10^{-4}}x(n - 1) + \brak{1.05 \times 10^{-3}}x(n - 2) \nonumber \\
		&+ \brak{1.76 \times 10^{-3}}x(n - 3) + \brak{1.76 \times 10^{-3}}x(n - 4) \nonumber \\
		&+ \brak{1.05 \times 10^{-3}}x(n - 5) + \brak{3.52 \times 10^{-4}}x(n - 6) \nonumber \\
		&+ \brak{5.02 \times 10^{-5}}x(n - 7)
	\end{align}
	From \eqref{eq:iir_filter_gen}, we see that the transfer function can be written as follows
	\begin{align}
		H(z) &= \frac{\sum_{k = 0}^{N}b(k)z^{-k}}{\sum_{k = 0}^{M}a(k)z^{-k}} \\
		&= \sum_{i}\frac{r(i)}{1 - p(i)z^{-1}} + \sum_{j}k(j)z^{-j}
		\label{eq:trans-func}
	\end{align}
	where $r(i)$, $p(i)$, are called residues and poles respectively of the partial 
	fraction expansion of $H(z)$. $k(i)$ are the coefficients of the direct polynomial 
	terms that might be left over. We can now take the inverse $z$-transform of
	\eqref{eq:trans-func} and get using \eqref{eq:anun},
	\begin{align}
		h(n) &= \sum_{i}r(i)[p(i)]^nu(n) + \sum_{j}k(j)\delta(n - j)
		\label{eq:h-n-expr}
	\end{align}
	Substituting the values,
	\begin{align}
		&h(n) = [\brak{2.76}\brak{0.55}^n \nonumber \\ 
		&+ \brak{-1.05-1.84\j}\brak{0.57+0.16\j}^n \nonumber \\
		&+ \brak{-1.05+1.84\j}\brak{0.57-0.16\j}^n \nonumber \\
		&+ \brak{-0.53+0.08\j}\brak{0.63+0.32\j}^n \nonumber \\
		&+ \brak{-0.53-0.08\j}\brak{0.63-0.32\j}^n \nonumber \\
		&+ \brak{0.20+0.004\j}\brak{0.75+0.47\j}^n \nonumber \\
		&+ \brak{0.20-0.004\j}\brak{0.75-0.47\j}^n]u(n) \nonumber \\
		&+ \brak{-6.81 \times 10^{-4}}\delta(n)
	\end{align}
	The values $r(i)$, $p(i)$, $k(i)$ and thus the impulse response function are computed and plotted at
	\begin{lstlisting}
		https://github.com/dhanushpittala11/EE3900-2022/blob/main/filter/codes/8_2_1.ipynb
	\end{lstlisting}
	The filter frequency response is plotted at
	\begin{lstlisting}
		https://github.com/dhanushpittala11/EE3900-2022/blob/main/filter/codes/8_2_2.ipynb
	\end{lstlisting}
	Observe that for a series $t_n = r^n$, $\frac{t_{n + 1}}{t_n} = r$.
	By the ratio test, $t_n$ converges if $|r| < 1$. We note that
	observe that $|p(i)| < 1$ and so, as $h(n)$ is the sum of convergent series,
	we see that $h(n)$ converges. From Fig. \eqref{fig:butter-imp}, it is clear
	that $h(n)$ is bounded. From \eqref{eq:z_trans},
	\begin{align}
		\sum_{n = 0}^{\infty}h(n) = H(1) = 1 < \infty
	\end{align}
	Therefore, the system is stable. From
	Fig. \eqref{fig:butter-imp}, $h(n)$ is negligible after $n \geq 64$, and we
	can apply a 64-bit FFT to get y(n). The following code uses the DFT matrix
	to generate $y(n)$ in Fig. \eqref{fig:butter-out}.
	\begin{lstlisting}
		https://github.com/dhanushpittala11/EE3900-2022/blob/main/filter/codes/8_2_3.ipynb
	\end{lstlisting}
	
	
	\begin{figure}[!htb]
		\includegraphics[width=\columnwidth]{/home/dhanush/Downloads/8_2_1.png}
		\caption{Plot of $h(n)$}
		\label{fig:butter-imp}
	\end{figure}
	
	\begin{figure}[!htb]
		\includegraphics[width=\columnwidth]{/home/dhanush/Downloads/8_2_2.png}
		\caption{Filter frequency response}
		\label{fig:butter-resp}
	\end{figure}
	
	\begin{figure}[!htb]
		\includegraphics[width=\columnwidth]{/home/dhanush/Downloads/8_2_3.png}
		\caption{Plot of $y(n)$}
		\label{fig:butter-out}
	\end{figure}
	
	\item What is the sampling frequency of the input signal?
	\\
	\solution run the following code to 
	Sampling frequency(fs)=44.1kHZ.
	\item
	What is type, order and  cutoff-frequency of the above butterworth filter
	\\
	\solution
	The given butterworth filter is low pass with order=2 and cutoff-frequency=4kHz.
	%
	\item
	Modifying the code with different input parameters and to get the best possible output.
	\solution a better filtering was found on changing the order of filter to 7.
	%
\end{enumerate}


\end{document}